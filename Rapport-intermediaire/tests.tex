\chapter{Résultats de tests}

\section{ L'experimentation de différents algorithmes parallèle sur ARM avec Android }

\paragraph{Les outils / Code parallèle \\}

	Il n'existe pas beaucoup d'outils pour paralléliser du code. Nous avons dû faire une recherche sur ce qui existe et utilisable sur Android (cf : 5-Besoins fonctionnels). \\

	Pour obtenir des gains de performance nous nous sommes penchés sur les librairies de parallèlisme classique. Parmi celles ci nos tests ont montrés qu'il était possible d'utiliser des technologies comme les Threads C++11, OpenMP et NEON pour la partie Android. En plus de ces technologies, les ordinateurs classiques nous permettent d'utiliser des technologies comme SSE, AVX, ou encore Cuda.\\

\paragraph{L'estimation des performances énergétiques \\}

	Le but est de tester la consommation énergetique du parallélisme sur Android et sur PC avec les différentes technologies exploitables. \\

	Nos tests consistent à exécuter des algorithmes où le parallélisme s'applique de façon optimale. Pour ce, des algorithmes où le parallélisme est maximal serait optimale, c'est-à-dire où chaque processeurs travailleraient sur des tailles de données équivalentes (cf : Non Fonctionnelle).
