\chapter{Analyse de l'existant}

\paragraph{L'état de l'art \\}
	De nombreux articles parlent de près ou de loin du sujet que nous allons traiter. \\
	La liste suivante est une liste non exhaustive des articles ainsi qu'un résumé de leurs contenus. \\

	\textit{"Parallelism Level Impact on Energy Consumption in Reconfigurable Devices" :}\cite{ref15}\\
		Impact de l’implémentation d'algorithmes parallèles sur la performance et la consommation énergétique. Il nous montre que la consommation énergétique diminue quand le parallélisme augmente.\\

	\textit{"Analytical Modeling And Simulation Of The Energy Consumption Of Independent Tasks" :}\cite{ref12}\\
		Comment la consommation d'energie peut être analysée, capturée par un modèle de consommation d'énergie analytique. Traite également de l'impact de la modification de la fréquence des processeurs sur la consomation d'énergie.\\

	\textit{"Modeling the Energy Consumption for Concurrent Executions of Parallel Tasks" :}\cite{ref14}\\
		Explore la consomation d'énergie des taches parralèles executé de façon concurentes. \\

	\textit {"On the Interplay of Dynamic Voltage Scaling and Dynamic Power Management in Real-Time Embedded Applications" :}\cite{ref13}\\
		Mise en evidence des differences entre deux techniques connues de gestion de l'energie pour les systèmes embarqués.\\

	\textit {"Prediction Models for Multi-dimensional Power-Performance Optimization on Many Cores" :}\cite{ref16}\\
		Présentation d'un modèle pour prédire les performances lorsque sont appliquées simultanement de multiples techniques de sauvegarde d'energie. \\

	\textit {"An Analysis of Efficient Multi-Core Global Power Management Policies: Maximizing Performance for a Given Power Budget" :}\cite{ref9} \\
		Ce papier évoque des techniques pour le contrôle et la gestion dynamique de l'énergie sur les systèmes multicoeurs.\\

	\textit {"On the Interplay of Parallelization, Program Performance, and Energy Consumption" :}\cite{ref10}\\
		Développe le lien etroit entre le parallelisme, la performane et la consomation énergetique d'un programme.\\

	\textit {"Hybrid MPI/OpenMP Power-Aware Computing" :}\cite{ref8}\\
		Présente un modèle hybrid MPI\footnote{MPI (The Message Passing Interface) est une norme définissant une bibliothèque de fonctions, utilisable avec les langages C, C++ et Fortran. Elle permet d'exploiter des ordinateurs distants ou multiprocesseur par passage de messages.}/OpenMP de programmation économe en energie.\\

	\textit {"Energy-Aware Computing for Android\footnote{Android est un système d'exploitation pour smartphones} Platforms" :}\cite{ref11}\\
		Présentation de la gestion de l'alimentation énergetique des systemes android.\\ 

	\textit {"Accurate Online Power Estimation and Automatic Battery Behavior Based Power Model Generation for Smartphones" : }\cite{ref7}\\
		Présentation de PowerBooster et PowerTutor, techniques de construction de modèle energetique pour smartphone, developpées par Google.\\

	\textit {"Extending Amdahl’s Law for Energy-Efficient Computing in the Many-Core Era":}\cite{ref6}\\
		Extension de la loi d'Amdahl\footnote{La loi d'Amdahl, énoncée par Gene Amdahl en 1967, exprime le gain de performance qu'on peut attendre d'un ordinateur en améliorant une composante de sa performance.} pour l'écoenergétique pour le développement multicoeurs.\\

	\textit {"Advanced Android Power Management and Implementation of Wakelocks":}\cite{ref5}\\
		Explique comment android gère l'utilisation batterie et pourquoi Linux à été choisie comme base pour android.\\

	\textit {"An Analysis of Power Consumption in a Smartphone":}\cite{ref4}\\
		Approche visant à améliorer la consomation et la gestion énergetique sur smartphone.\\

	\textit {"Android Power Management: Current and Future Trends":}\cite{ref3}\\
		Présente de nombreuses recherches traitant de la diminution de la consomation énergetique \\

	\textit {"Energy Consumption Modeling for Hybrid Computing" :}\cite{ref2}\\
		Montre qu'un programme utilisant le parallelisme utilise moins d'énergie qu'un programme non parallèle.\\

	\textit {"Power-Aware Speedup" :}\cite{ref1}\\
		Présentation d'un modèle permettant de juger de la performance mais aussi de la consomation énergetique des algorithmes parallèles\\