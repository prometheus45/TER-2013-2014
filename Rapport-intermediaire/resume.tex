\chapter{Résumé du projet}

Les machines actuelles, du smartphone aux supercalculateurs, possèdent des
processeurs multi-coeurs. Afin d'augmenter les performances il faut donc
paralléliser les codes. Cependant la parallélisation n'est pas très répandue dans
les périphériques nomades dont le nombre de cœurs sert généralement à
exécuter plusieurs tâches distinctes et à diminuer la consommation
énergétique en modifiant les fréquences et en désactivant les processeurs
inutilisés. Pour exécuter des calculs peut intensifs, on peut donc se poser la
question de : soit faire fonctionner un seul processeur à sa fréquence
maximale, soit les paralléliser et utiliser plusieurs cœurs à fréquence réduite.
Une comparaison entre des codes écrits en Java et C++ pour Android pourra
également être envisagée afin d'étudier l'impact du langage sur les
performances/consommation.