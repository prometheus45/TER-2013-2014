\chapter{Résumé du projet}

Notre projet, proposé par M. JUBERTIE, nous entraîne dans l'étude de l'impact de la parallélisation d'algorithmes sur la consommation énergétique des architectures parallèles.\\ 

Les machines actuelles, du smartphone aux supercalculateurs, possèdent des 
processeurs multi-c\oe{}urs. Afin d'augmenter les performances le développeur doit s'appliquer à développer des codes s'exécutant en parallèle. Cependant, la parallélisation n'est pas très répandue dans les périphériques nomades. Les nombreux c\oe{}urs servent généralement à exécuter plusieurs tâches distinctes. \\

Pour juger de l'impact du parallélisme sur consommation énergétique, certains choix s'offrent à nous  : soit faire fonctionner un seul processeur à sa fréquence 
maximale, soit utiliser plusieurs c\oe{}urs à fréquence réduite. \\

Pour cela de nombreux algorithmes parallèles seront implémentés et exécutés sur différentes architectures.
Une comparaison entre des codes écrits en Java et C/C++ pour Android pourra 
également être envisagée afin d'étudier l'impact du langage sur les 
performances et la consommation. \\ 
