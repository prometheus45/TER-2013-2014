\chapter{Résumé du projet}

Notre projet, proposé par M. JUBERTIE, nous entraîne dans l'étude de l'impact de la parallélisation d'algorithmes sur la consommation énergétique des architectures parallèles et les contextes d'utilisations.\\ 

Les machines actuelles, du smartphone aux supercalculateurs, possèdent des 
processeurs multi-c\oe{}urs. Afin d'augmenter les performances il faut donc 
paralléliser les codes. Cependant, la parallélisation n'est pas très répandue dans les périphériques nomades dont le nombre de c\oe{}urs sert généralement à exécuter plusieurs tâches distinctes et à diminuer la consommation 
énergétique en modifiant les fréquences et en désactivant les processeurs 
inutilisés. Pour exécuter des calculs peut intensifs, on peut donc se poser la question de : soit faire fonctionner un seul processeur à sa fréquence 
maximale, soit les paralléliser et utiliser plusieurs c\oe{}urs à fréquence réduite. \\

Pour cela de nombreux tests basés sur des calculs en parallèle seront implémentés et exécutés sur différentes architectures ; et ainsi obtenir un début de réponse à notre problématique.
Une comparaison entre des codes écrits en Java et C++ pour Android pourra 
également être envisagée afin d'étudier l'impact du langage sur les 
performances/consommation. \\ 
