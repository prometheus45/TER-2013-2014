\chapter{Résumé du projet}

Notre projet est proposé par M. JUBERTY et nous entraine dans l'étude des performances de la parallélisation sur plusieurs systèmes et processeurs différents. Aujourd'hui, les machines actuelles, du smartphones aux supercalculateurs, possèdent des processeurs multi-cœurs. Pour de meilleurs performances, il faut paralléliser les codes. 
\\	Mais cette pratique n'est pas répandu sur les architectures nomades dont le nombre de cœur sert généralement à exécuter plusieurs tâches distincts et à diminuer la consommation énergétique en modifiant les fréquences et en désactivant les processeurs inutilisés. Notre but à l'issue du projet est d'établir de qu'elle manière exploiter le potentiel des processeurs nomades avec des méthodes de parallélisation tout en conservant une faible consommation énergétique. Faut-il plutôt utiliser un seul cœur avec une fréquence élevée ou plutôt utiliser plusieurs cœurs avec des fréquences plus faibles ?
       